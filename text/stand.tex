\chapter{Stand der Forschung}
Dieses Kapitel beschäftigt sich mit der vorhandenen Fachliteratur bezüglich der beschriebenen Problemstellung.
\section{\textit{„block“} und \emph{„blocking problem“}}
Die Optimierung des Güterverkehrssystems interessiert die Forscher schon lange. Die beschriebene Problemstellung ist in der Literatur unter dem englischen Begriff bzw. Stichwort \emph{„blocking problem“} zu recherchieren. In Bezug auf den technischen Aspekt sowie die Modellierung unterscheidet sich diese Arbeit aber von vielen Aufsätzen durch die Definition von \textit{„block“}. Bei der Erstellung einer Blockkante zwischen zwei Anlagen wird in vielen Aufsätzen eine direkte Verbindung zwischen den beiden Stationen hergestellt. Beispielsweise die kontinuierliche Entscheidungsvariable  \(x_{ijkp}\)  in \cite{bodin1980model} deutet darauf hin, wie viele Eisenbahnwaggons vom Bahnhof \(i\) im \(j\) umgestellt werden, welche durch \(p.\) Pfad an den Bahnhof \(k\) landet. Ein Wagon im Bahnhof $i$, das den Bahnhof $k$ als Ziel-Knoten hat, wird also als nächstes zur Umstellung in dieser Blockkante zum Bahnhof $j$ transportiert. Die Blockkante vom $i$ bis zum $j$ ist eigentlich ein Bestandteil der Sendungsroue $p$, die an $k$ endet. Im Allgemeinen verlässt bei $j=k$ ein Wagon das Optimierungssystem.\\
Im Gegensatz dazu und angepasst an die Voraussetzungen der Deutschen Bahn, bedeutet eine Blockkante bei \cite{homfeld2012consolidating} eine Gruppe der Waggons, die in Züge aufgeteilt wird. Die Züge auf eine Blockkante haben zwei Kapazitätsrestriktionen (nach dem Gewicht und der Länge). Das Einsetzen von mehr als einem Zug pro Kante ist außerdem auch erlaubt \textit{(vgl. \cite{homfeld2012consolidating})}. In dieser Arbeit wurde Homfelds Konzept der Blockkante so erweitert, dass die Kanten die konkrete Zeitkomponenten bzw. Abfahrt- und Ankunftszeit sowie Wartezeit auch explizit aufweisen.\\

\section{„Fluss-Model“ und „Pfad-Model“}
Allgemein kann jedes \textit{„blocking problem“} als ein Grundmodel \textit{„Multi-commodity capacitated network design problem“} (MCNDP) mit zusätzlichen Entscheidungsvariablen sowie Nebenbedingungen betrachtet werden. Das Model lässt sich mathematisch mit zwei Formulierungen, nämlich das „Fluss-Model“ \textit{(auf engl. arc-based)} und das „Pfad-Model“ \textit{(auf engl. path-based)} , interpretieren. Für Details zu mathematischen Formulierungen wird auf die Aufsätze \cite{hewitt2010combining} und \cite{frangioni20090} verwiesen. Die beschriebene Problemstellung hat dementsprechend auch zwei Formulierungen. \cite{homfeld2012consolidating} hat die Formulierungen sowie das Dual-Problem und Ungleichungen in Bezug auf das Hierarchie-Model aufgebaut.\\

\section{In der Literatur vorgeschlagene Lösungsansätze}
Für das {„blocking problem“} sowie das Grundmodel \textit{„Multi-Commodity capacitated Network design Problem“} wurden in der Literatur verschiedene exakte bzw. heuristische Lösungsansätze vorgeschlagen. Wegen der exponentiell steigenden Anzahl der Entscheidungsvariablen bei der Pfad-Formulierung wurden vor allem die auf \textit{„Column Generation“} basierende Methoden angewendet. \cite{barnhart2000railroad} bietet einen heuristischen Ansatz basierend auf \emph{„lagrangian relaxation“} für das als Fluss-Model formulierte blocking-Problem an. Der Zerlegungsverfahren teilt das komplexe MIP-Model in zwei Unterprobleme auf. Einige Ungleichungen werden zu dem Teilproblem hinzugefügt, um die unteren Schranken zu verschärfen und die Erzeugung der Lösungen zu erleichtern. Die Subgradienten-Optimierung wird auch zur Lösung des Lagrangian Dual-Problems verwendet.\\
\cite{hasany2018modeling} haben ebenfalls eine Heuristik basierend auf dem Dekomposition-Algorithmus entwickelt, der das linearisierte Modell in zwei einfachere und unabhängige Unterprobleme unterteilt. Das erste Unterproblem zielt darauf ab, die beste Route für die Güterwagen zu finden. Die Route erfüllt die Fluss- sowie Nachfragebeschränkungen. Das zweite aber ermittelt für den Pfad, den das erste Unterproblem erzeugt hat, die optimale Anzahl der durch einzelne Blockkanten fahrenden Züge. Weil manche Restriktionen durch Zerlegungsverfahren relaxiert werden, kann dieser Ansatz die Zulässigkeit der generierten Lösung nicht gewährleisten.\\
\cite{crainic2000simplex} schlagen eine Simplex-basierte Tabu-Suchmethode für das MCNDP vor, indem sie eine auf dem Pfad-Model basierende Problemformulierung verwenden. Ihre Methode kombiniert die Spaltengenerierung mit Pivot-artigen Bewegungen einzelner Sendungsflüsse, um im Endeffekt die Pfadflussvariablen zu bestimmen.\\
\cite{ghamlouche2003cycle} stellen eine zyklusbasierte Nachbarschaft zur Anwendung in einem metaheuristischen Lösungsansatz für MCNDP dar. Die Hauptidee der zyklusbasierten lokalen Verschiebungen steckt in der Umleitung der Güterflüsse in Zyklen, um vorhandene Kanten aus dem Netzwerk zu entfernen und durch neue Kanten zu ersetzen. In \cite{ghamlouche2004path} erweitern sie die Anwendung der neuen Nachbarschaft in einem evolutionären Algorithmus. Ihr Rahmenkonzept basiert auf \emph{„Pfad-Relinking“}, was bezüglich der \emph{„Scatter-Suche“} für evolutionäre Algorithmen ursprünglich in \cite{glover1998template} vorgeschlagen wurde. Die zyklusbasierte Nachbarschaft erzeugt eine Elite-Kandidatenmenge der Lösungen durch einen Tabu-Suchalgorithmus, welche als Rekombinationsmethode zur Erzeugung der Nachkommen funktioniert. Bei Aktualisierung des Lösungspools wird die Unähnlichkeit der Lösungen als zusätzliche Komponente bei der Berechnung des Zielfunktionswerts betrachtet.\\
\cite{alvarez2005scatter} schlagen ebenfalls eine Version des Scatter-Suchalgorithmus für MCNDP vor, indem der von \cite{feo1995greedy} vorgestelltem \emph{GRASP} \emph{(auf engl. greedy randomized adaptive search procedure)} zur Herstellung eines diversifizierten initialen Lösungspools angewendet wurde. Beim nächsten Schritt werden verschiedene kombinierte Untermengen aus den Sendungspfaden im Lösungspool erstellt. Dann aus jeder Untermenge werden die besten Pfade für ein Verbesserungsverfahren ausgewählt. Ein Zulässigkeitsmechanismus ist auch zuständig, um die unzulässig generierten Lösungen wiederherzustellen. Im Gegensatz dazu betrachtet die Scatter-Suche von \cite{paraskevopoulos2016cycle} die Sendungspfade zur Rekombination der Lösungen zum Erzeugen der Nachkommen nicht. Stattdessen werden die Kanten aus einer Untermenge der Lösungen kombiniert. Sie beanspruchen, dass diese Modifikation die Leistung des Scatter-Suchalgorithmus verbessert, weil damit mehr Varianten durchsucht werden können, was zu einem diversifizierten stärkeren Pool der Nachkommen führt. Eine iterative Methode zur lokalen Suche ist dann für die Verbesserung einzelner der Nachkommen zuständig, indem  eine Heuristik die binären Variablen bezüglich der Kantenbelegung festsetzt, und dann ein MIP-Model zur Bestimmung der Flussvariablen oder Unzulässigkeit der festen Kanten aufgebaut wird. Ein zyklusbasierter Nachbarschaft-Operator ermöglicht vollständige oder teilweise Umleitung mehrerer Güter. 

