\chapter{Konzept}

\section{Subgradientenverfahren}
Das \emph{Subgradientenverfahren} wurde von \cite{HeKa71} \todo{verify} vorgeschlagen um das duale Lagrange Problem (LDP) approximativ zu lösen. Das Verfahren ist iterativ, dabei wird in jeder Iteration zuerst das Lagrange Problem $Z_D(\mu)$ für gegebene Multiplikatoren $\mu$ gelöst und im zweiten Schritt neue Multiplikatoren für die nächste Iteration berechnet. Das ganze wird wiederholt bis eine Abbruchbedingung erfüllt ist.
\begin{algorithm}
	\label{subgradientenverfahren}
	\caption{Das Subgradientenverfahren}
	Initialisiere $\mu^0$\\
	Iterationszähler $k$ = 0\\
	Berechne obere Schranke $Z^*$ (evtl. heuristisch)\\	
	\While{Keine Abbruchbedingung erfüllt} {
		Löse Lagrange Relaxation $Z_D(\mu^k), x^k$ ist die optimale Lösung\\
		Berechne Subgradient-Vektor $s^t = b - Ax^k$\\
		Berechne Suchrichtung $d^k$ \\
		Berechne Schrittweite $t^k$ \\
		Aktualisiere Multiplikatoren $\mu^{k+1} = \mu^k + t^k d^k$\\
		$k = k + 1$\\
	}
\end{algorithm}

Algorithmus \ref{subgradientenverfahren} beschreibt die Grundstruktur des Verfahrens. Die entscheidenden Faktoren, die die Performanz des Verfahrens und die Lösungsqualität bestimmen, sind die Berechnung der Suchrichtung $d^k$ und der Schrittweite $t^k$ sowie die Abbruchbedingungen. Diese Aspekte sollen im weiteren genauer erläutert werden.

\section{departure Time heuristic adjuster}



\(ref\) Kantenergebnisse-Referenz

\begin{algorithm}
	\label{Abfahrtszeit-Einsteller}
	\caption{Heuristischer Abfahrtszeit-Einsteller}
	\KwIn{die Pfad-Kombinationen mit frühesten Abfahrtzeiten\((k_i , p_i^{t_{min}})\)}
	\KwOut{optimaler Routenplan \(RP(k_i , p_i^{t^*})\)}
	$ RP \gets$ initialisiere mit \((k_i , p_i^{t_{min}})\)\\
	\For{jede Kombination \(k_i , p_i\) aus Input}{$ ref \gets Aggregation(k_i , p_i^{t_{min}})$ }
	\While{\(ref\) enthält eine Kante mit Verschiebungspotential}
	{
    	$L \gets$ initialisiere eine neue Verschiebungsiste\\
	    Kandidatenkante \(er_{ij}^{t_m} \gets\) erste Kante mit Verschiebungspotential\\auf \(t_n\) (maximal vorkommende Abfahrtzeit) aus \(ref\)\\
        \If{\(e_{ij}^{t_n} \) ist für die durchfahrende Sendungen auf  \(e_{ij}^{t_m} \) zulässig}{
        	\(k_i \gets\) die durchfahrende Sendungen auf  \(e_{ij}^{t_m} \)\\
            aktualisiere VerschiebungsListe \(L \gets k_i , e_{ij}^{t_m}  , t_n\)\\
            \(Tiefensuche Der Sendungspfade( L )\)\\
            \(VerschiebungUndEvaluierung(L)\)\\
       }  
	}
\end{algorithm}

\begin{algorithm}
	\label{TiefensuchePfad}
	\caption{Tiefensuche Der Sendungspfade}
	\KwIn{Untermenge der Verschiebungsliste (neue Verschiebungsaufträge) $L_n$}
	\For{jede Verschiebung \(k_i , e_{ij}^{t_m}  , t_n\) aus $L_n$}
	{
	    $p_k \gets$ aktuell optimaler Pfad der Sendung $k_i$ aus $RP$\\
        $t_i^n \gets$ maximal zulässige Abfahrtzeit aller Kanten der $p_k$ nach $e_{ij}^{t_n}$\\
	    \For{jede Kante $e_{lm}^{t_r}$ aus $p_k$}
	    {
	        \If{$e_{lm}$ ist $e_{ij}$}{setze fort}
	        
	        $er_{lm}^{t_r} \gets$ hole das gemappte Kantenergebnis aus $ref$\\
	        
	        \If{versuche alle durchfahrende Sendungen in $er_{lm}^{t_r}$ auf ein spätere Zeitpunkt bis zum maximal $t_l^n$ zu verschieben}
	            {
	                \If{neue Verschiebungszeitpunkt $< t_l^n$}
	                {initialisiere die neue Verschiebungsliste}
	                
	                $L_n \gets$ neue Verschiebungsaufträge\\
	                $Tiefensuche Der Sendungspfade( L_n )$\\
	            }
	    }
	}
\end{algorithm}

versuche alle durchfahrende Sendungen in $er_{lm}^{t_r}$ auf ein spätere Zeitpunkt bis zum maximal $t_l^n$ zu verschieben