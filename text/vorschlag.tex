\chapter{Lösungsansatzvorschlag}
Hinsichtlich des mathematischen Pfad-Models besteht die Optimalität in der besten Kombination aus den möglichen Pfaden. Daher wird wegen untrennbarer Sendungen jeder Sendung ein bloßer Pfad so zugeordnet, dass der Routenplan aus den Pfaden die minimalen Kosten bezüglich der entgegengesetzten Ziele (wie Schnelligkeit und Bündelungseffekt) aufweist.\\
Wegen der exponentiell steigenden Anzahl der Pfad-Variablen wurde in der Literatur oft die auf Spaltengenerierung basierenden Methoden angewendet. wenn das Szenario also \(n\) Sendungen umfasst, und die Sendung \(i\) eine Anzahl von \(k_i\) Pfade besitzt, hat der Lösungsraum $\prod\limits_{i=1}^n k_{i}$ verschiedene Möglichkeiten. Trotzdem stehen nur \(n\) Pfade in der Basis jeder Simplex-Tableau (Lösung). Deshalb kann man zuerst durch eine Initial-Lösung viele Pfade vernachlässigen, und dann iterativ die Pfade (Spalten) mit der potenziellen Verbesserung der Basis hinzufügen.\\
\cite{homfeld2012consolidating} stellt zwar Anhand der mathematischen Formulierung des dualen Problems eine Formel für den Pricing-Schritt der Spaltengenerierung vor, die mit den aktuellen dualen Variablen sowie den Parameter bezüglich der Kanten und Anlagen verbunden ist, aber ist es im Vergleich zu der Pricing-Formel eines klassischen \emph{Multi-Commodity flow model} komplexer, da die Blockprobleme prinzipiell mehr Restriktionen und entsprechend mehr duale Variablen beinhalten. Aus diesem Grund versuchen wir in dieser Arbeit, statt der expliziten Rechnung der dualen Variablen und der daraus gerechneten reduzierten Werten die kompetitiven Pfade durch Effizienz-Verhältnisse wie Zugauslastung zu ermitteln.\\
Mit der Inspiration des Dekomposierungsverfahrens zerlegen wir das Problem in zwei einfachere Unterprobleme. Wir versuchen dabei erstmals, aus den günstigeren Einzel-Pfaden (in Bezug auf Distanz bzw. Reisezeit) eine Kombination zu bestimmen. Dafür bauen wir einen evolutionären Algorithmus anhand des Rahmenkonzepts der Scatter-Suche auf. Da die durch Preprocessing generierten Pfade aus den Kanten mit den frühesten Abfahrtzeiten bestanden sind, soll im zweiten Schritt die Abfahrtzeit der Kanten aus den festen Pfaden, die durch erstes Unterproblem erzeugt worden sind, angepasst werden. \\

\section{Preprocessing-Schritt}
Das Ziel des Preprocessingsverfahrens besteht darin, alle zulässigen Pfade jeder Sendung zu erzeugen, die als Input zur Erstellung der Entscheidungsvariablen eines Pfad-basierten Algorithmus einzugeben sind. Der betroffene Graph des Szenarios sowie die Liste aller Sendungen sind ursprünglich diesem Schritt als Input einzugeben. Es gibt drei Einschränkungen, die die Zulässigkeit eines Pfades bestimmen. Strukturbeschränkung ist zuerst zu nennen, d.h. je nach Graph, ob die Reihenfolge der Knoten und Kanten bzw. der Pfad zulässig ist. Die maximale Transportzeit sowie die maximale Anzahl der Umstellungen sind auch für die Zulässigkeit eines Pfades entscheidend. Sie beziehen sich auf das Input bezüglich der Sendungen.\\

Die klassische Tiefensuche \emph{(auf engl. Depth first search algorithm)} ist ein Algorithmus zum Durchqueren bzw. Durchsuchen der Graph- oder Baum-Datenstrukturen. Die Anwendung dieses Algoritmus ist jedoch mit Herausforderungen verbunden. Einerseits zur Überprüfung der Zulässigkeit eines Pfades nach der maximalen Transportzeit soll der Zeit expandierte Graph durchsucht werden, da dieser Graph die konkreten Abfahrtzeiten und die daraus berechneten Wartezeiten aufweist. Bei den Szenarien aus dem Schienenverkehr sind normalerweise die Wartezeit im Vergleich zu der Transferzeit (reiner Transfer im Zug) besonders dominant. Die Überprüfung der Zulässigkeit nach der Transferzeit allein, was nach dem räumlichen Graphen ebenfalls zu ermitteln ist, ist daher nicht versprechend. Auf der anderen Seite ist aber die Suche durch den Zeit expandierten Graphen wegen der Komplexität sehr viel rechenaufwändiger als beim räumlichen Graphen.\\

Als Lösung schlägt diese Arbeit eine spezielle Version der Tiefensuche vor, die ein vereinfachtes Exemplar des räumlichen Graphen nach den strukturell zulässigen Pfaden sucht. Zur Überprüfung der Zulässigkeit nach der maximalen Transportzeit wird trotzdem der expandierte Graph nach der Kante mit der frühesten zulässigen Abfahrtzeit durchsucht. Somit erreicht die Tiefensuche im Endeffekt einen Pfad aus den Kanten mit den frühesten Abfahrtzeiten. Dann wird die Zeitspanne zwischen der Ankunftszeit am Ziel für diesen Pfad (die früheste Ankunftszeit basiert auf Transport mit diesem Pfad) und der spätesten Ankunftszeit (basiert auf maximaler Transportzeit der Sendung) gerechnet. 
Die späteren aber noch zulässigen Zeitlichen Kanten lassen sich diese Zeitspanne entsprechend auch generieren. Anhand eines rekursiven Aufrufs läuft der Algorithmus weiter, sobald er alle zulässigen Pfade herausfindet.

\section{Heuristisches Verfahren zur Einstellung der Abfahrtzeiten}
Jede Pfad-Kombination besteht aus den Pfaden mit frühesten Abfahrtzeiten einzelner Sendung, welche die Tiefensuche durch den Preprocessing-Schritt erzeugt.  Die Tiefensuche stellt auch sicher, dass alle Pfade bezüglich der maximalen Transportzeit und maximallen Anzahl der Umstellungen zulässig sind, und generiert auch die noch zulässige Verschiebungsmöglichkeiten für jede Kante einzelnes Pfades. Trotzdem ist zur Erstellung eines Routenplans aus der Kombination die Bestimmung der Abfahrtzeitzeit der Kanten ebenfalls notwendig. Sonst würden die Sendungen trotz der gleichen räumlichen Kanten schließlich nicht gebündelt. Aus diesem Grund würden die Abfahrtzeit der Kanten aus festen Pfaden hier eingestellt, d.h. die Frage im diesen Unterproblem besteht darin, ob es sich lohnt, eine \(A\to B\) Kante eines Sendungspfads zu verschieben, damit die Sendung mit anderen Sendungen, deren Pfad auch die \(A\to B\) Kante mit verschiedener Abfahrtzeit umfasst, konsolidiert wird. In anderen Worten formuliert, soll zwischen die Ersparnis durch Bündelungseffekt und die schnellere Routen mit gleichen räumlichen Knoten und Kanten ein Kompromiss festgesetzt werden.\\



\section{Evolutionärer Algorithmus zur Optimierung der Pfad-Kombination}
Anhand einer Scatter-Suche und Rekombination-Methode versuchen wir die Vielfältigkeit bzw. Exploration zu erhöhen. Im Gegensatz dazu fokussiert eine heuristische lokale Nachbarschaft-Suche auf das lokale Optimum, und versucht jede Lösung aus dem Pool unter einem gewissen Nachbarschaft zu verbessern. Für Rekombinationsmethode kan die methode \emph{"path relinking"} vorgeschtelt von \cite{glover2003scatter} angewendet werden.\\
\todo{Read scatter search and path relinking}