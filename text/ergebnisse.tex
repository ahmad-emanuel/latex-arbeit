\chapter{Ergebnisse}

\begin{table}[h]
\begin{tabular}{lccccccc}

{\bf Problem} & {\bf Dienste} & {\bf Busse} &   {\bf LB} &   {\bf UB} & {\bf Zeit} & {\bf Spalten CSP} & {\bf Fixierung} \\
\hline
    320A01 &         54 &         25 &     79.333 &     86.418 &   05:32:44 &    100.000 &            \\

    320A03 &         54 &         24 &     79.178 &     86.184 &   07:03:14 &    120.586 &            \\

    320A05 &         40 &         20 &     62.981 &     66.166 &   06:02:23 &    100.000 &            \\

    320A07 &         58 &         27 &     85.927 &     92.968 &   06:26:15 &    178.231 &            \\

    320A09 &         54 &         25 &     81.758 &     87.568 &   06:10:01 &    167.585 &            \\
\hline
           &   {\bf 52} & {\bf 24,2} & {\bf 77.835} & {\bf 83.861} & {\bf 06:14:55} & {\bf 133.280} &     {\bf } \\

\end{tabular}
\caption{Verfahren von Huisman für ausgewählte ECOPT-Instanzen mit 4 Depots und 320 Servicefahrten, Variante A}.
\end{table}
